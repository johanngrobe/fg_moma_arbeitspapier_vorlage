% !TeX root = Arbeitspapiervorlage.tex
\chapter{Einleitung}\label{ch_einleitung}

So kann man zitieren \parencites[17]{hesse2018}[239]{zimmermann2023}. Das ist kurisver Text \textit{Lebenswerte Städte und Gemeinden}.

So kann man mehrere Kurzverweise machen\parencites[29]{birk2020}{bbsr2019}{deutscherstaedtetag2018}{uba2022}[246]{zimmermann2023}. 

Das ist eine einfache Quellenangabe \parencite[13]{schwedes2013a}. 


\section{Ausgangssituation}

Hier ist ein Beispiel für ff. \parencite[96\psq]{piesold2021}. 


\section{Zielsetzung und Forschungsfragen}\label{ch_zielsetzung}


Das ist eine Aufzählung
\begin{itemize}
    \item Nummer 1
    \item Nummer 2
    \item Nummer 3
    \item Nummer 4
\end{itemize}

Hier werden Querreferenzierungen gemacht \autoref{ch_ergebnisse}, \autoref{ch_diskussion}.